\documentclass{article}

\DeclareRobustCommand\package[1]{\texttt{#1}}

\title{\package{ebproof} -- proof trees}
\author{Emmanuel Beffara}

\usepackage{color}
\usepackage{fancyvrb}
\usepackage{hyperref}

\usepackage{ebproof}

\newcommand\lit[1]{\texttt{#1}}
\newcommand\cs[1]{\lit{\char`\\#1}}
\newcommand\env[1]{\lit{#1}}
\newcommand\opt[1]{\lit{#1}}
\newcommand\meta[1]{$\langle$\textit{#1}$\rangle$}
\newcommand\oarg[1]{\lit{[}\meta{#1}\lit{]}}
\newcommand\marg[1]{\lit{\{}\meta{#1}\lit{\}}}

\newenvironment{csdoc}[1]{%
  \begin{flushleft}#1\end{flushleft}%
  \begin{quote}%
}{%
  \end{quote}%
}

\newenvironment{example}{%
  \VerbatimOut{example.tex}}{%
  \endVerbatimOut
  \par\noindent Code:
  \begin{quote}
    \VerbatimInput{example.tex}
  \end{quote}
  Result:
  \begin{quote}
    \input{example.tex}
  \end{quote}
}

\begin{document}
\maketitle

\tableofcontents

\section{Introduction}

The \env{ebproof} package provides commands to typeset proof trees, in the
style of sequent calculus and related systems.
The structure is inspired by the
\href{http://math.ucsd.edu/~sbuss/ResearchWeb/bussproofs/}{\package{bussproofs}}
package, in particular proofs are built using statements that operate on a
stack of sub-proofs, in postfix notation.

\section{Usage}

\subsection{Package loading}

\begin{csdoc}{\cs{usepackage}\oarg{options}\lit{\{ebproof\}}}
  Load the package.
  The \meta{options} provide default formatting options, as described in
  section~\ref{sec:options}.
  Specific options may also be specified for each proof tree and for each
  statement in a proof tree.
\end{csdoc}

\subsection{Environments}

\begin{csdoc}{
    \cs{begin}\lit{\{prooftree\}}\oarg{options}
      \meta{statements}
    \cs{end}\lit{\{prooftree\}}}
  Typeset the proof tree desribed by the \meta{statements}, as described in
  section~\ref{sec:statements}.
  The \meta{options} provide default formatting options for the proof tree.
  This environment can be used either in math mode or in text mode.
  It produces a proof tree at the current position in the text flow.
\end{csdoc}

\begin{csdoc}{
    \cs{begin}\lit{\{prooftree*\}}\oarg{options}
      \meta{statements}
    \cs{end}\lit{\{prooftree*\}}}
  Typeset the proof centered on a line of its own; it is essentially
  equivalent to wrapping the \env{prooftree} environment inside a \env{center}
  environment.
\end{csdoc}

\subsection{Statements}
\label{sec:statements}

The basic statements for building proofs are:
\begin{csdoc}{\cs{Hypo}\oarg{options}\marg{text}}
  Make a hypothesis, i.e.
  a leaf in the proof tree, with no line above.
\end{csdoc}
\begin{csdoc}{\cs{Infer}\oarg{options}\marg{arity}\oarg{label}\marg{text}}
  Build an inference step.
  The \meta{arity} is the number of sub-proofs, it may be any number
  including 0 (in this case there will be a line above the conclusion but no
  sub-proof).
  If \meta{label} is present, it is used as the label on the right of the
  inference line.
  The \meta{text} is the conclusion of the rule.
\end{csdoc}
\begin{csdoc}{\cs{Ellipsis}\marg{label}\marg{text}}
  Typeset vertical dots, with a label on the right, and a new conclusion.
  No inference lines are inserted.
\end{csdoc}

Each proof tree has a vertical axis, used for alignment of successive steps.
The position of the axis is deduced from the text of the conclusion in the
above commands.
If \meta{text} contains the alignment character \verb|&| then the axis is
set at that position, otherwise the axis is set at the center of the
conclusion text.
For instance, successive statements like \verb|\Infer1{\Gamma&\vdash\Delta}|
will produce conclusions aligned on the \cs{vdash} symbol.

The following additional statements may be used to affect the format of the
last proof tree on the stack:

\begin{csdoc}{\cs{Alter}\marg{commands}}
  Modify the proof with arbitrary commands, assuming that these commands do
  not affect the size.
  A typical use is \verb|\Alter{\color{red}}|, this will typeset the subproof
  in red.
\end{csdoc}
\begin{csdoc}{\cs{Delims}\marg{left}\marg{right}}
  Put left and right delimiters around the whole sub-proof, without changing
  the alignment (the spacing is affected by the delimiters, however).
  The \meta{left} text must contain an opening occurrence of \cs{left} and the
  \meta{right} text must contain an occurrence of matching occurrence of
  \cs{right}.
  For instance, \verb|\Delims{\left(}{\right)}| will put the
  sub-proof between parentheses.
\end{csdoc}

\subsection{Options}
\label{sec:options}

The formatting of trees, conclusion texts and inference rules is affected by
options, specfied using the syntax of \package{xkeyval}.
They can be set for a whole tree or a single statement, or even globally using
\cs{setkeys}\lit{\{ebproof\}}\marg{options}.

\subsubsection{General shape}

\begin{csdoc}{\opt{updown=}\meta{boolean}}
  If set to \lit{true}, the proofs are written upside down, with the
  conclusion of rules set above and the premisses below.
\end{csdoc}
\begin{csdoc}{\opt{center=}\meta{boolean}}
  If set to \lit{true}, the tree produced by the \env{prooftree} environment
  will be vertically centered around the text line.
  If set to \lit{false}, the base line of the tree will be the base line of
  the conclusion.
  The default value is \lit{true}.
\end{csdoc}

\subsubsection{Format of inference rules}

\begin{csdoc}{\opt{rulestyle=}\meta{name}}
  Defines the style of rules, among the following choices:
  \begin{itemize}
  \item \opt{none}: no inference line at all
  \item \opt{simple}: a single line (this is the default style)
  \item \opt{double}: two lines
  \item \opt{dashed}: a dashed line
  \end{itemize}
\end{csdoc}
\begin{csdoc}{
    \opt{leftlabel=}\meta{text} \\
    \opt{rightlabel=}\meta{text}}
  The text to use as the labels of the rules, on the left and on the right
  of the inference line.
  Using the second optional argument in \cs{Infer} is equivalent to setting
  the \env{rightlabel} option with the value of that argument.
\end{csdoc}
\begin{csdoc}{
    \opt{template=}\meta{macro} \\
    \opt{lefttemplate=}\meta{macro} \\
    \opt{righttemplate=}\meta{macro}}
  Defines how conclusions are formatted.
  The macros are arbitrary \TeX\ code, composed in horizontal mode, with the
  macro argument \verb|#1| standing for the actual text passed to the
  \cs{Hypo} and \cs{Infer} statements.
  The \opt{template} value is used for conclusions with no alignment mark.
  The \opt{lefttemplate} and \opt{righttemplate} values are used on the left
  and right side of the alignment mark when it is present.
  The default value for \opt{template} is simply \verb|$#1$|, so that
  conclusions are set in math mode.
  The default values for \opt{lefttemplate} and \opt{righttemplate} are
  similar, with spacing assuming that a relation symbol is put near the
  alignment mark, so that \verb|\Infer1{A &\vdash B}| is spaced correctly.
\end{csdoc}
\begin{csdoc}{
    \opt{leftlabeltemplate=}\meta{macro} \\
    \opt{rightlabeltemplate=}\meta{macro}}
  These macros are used to typeset the text of labels on the left and right of
  inference lines.
  The default values are \verb|#1| so that labels are set in plain text mode.
\end{csdoc}

\subsubsection{Dimensions}

\begin{csdoc}{\opt{hsep=}\meta{dimension}}
  The horizontal separation between sub-proofs in an inference.
\end{csdoc}
\begin{csdoc}{\opt{rulemargin=}\meta{dimension}}
  The spacing above and below inference lines.
\end{csdoc}
\begin{csdoc}{\opt{rulesep=}\meta{dimension}}
  The spacing between lines in double-line inferences.
\end{csdoc}
\begin{csdoc}{\opt{thickness=}\meta{dimension}}
  The thickness of inference lines.
\end{csdoc}
\begin{csdoc}{\opt{labelsep=}\meta{dimension}}
  The spacing between an inference lines and its labels.
\end{csdoc}

\section{Examples}

\subsection{Simple proof tree}

\begin{example}
\begin{prooftree}
  \Hypo{ \Gamma, A &\vdash B }
  \Infer1{ \Gamma &\vdash A\to B }
  \Hypo{ \Gamma \vdash A }
  \Infer2{ \Gamma \vdash B }
\end{prooftree}
\end{example}

\subsection{Templates and alignment}

\begin{example}
\begin{prooftree}[rulestyle=dashed,righttemplate={,#1}]
  \Hypo{ABC}
  \Infer1{&ABC}
  \Infer1{A&BC}
  \Infer1{AB&C}
  \Infer1{ABC&}
\end{prooftree}
\end{example}

\subsection{Fancy formatting}

\begin{example}
\begin{prooftree}
  \Hypo{A}
  \Ellipsis{$\pi_1$}{B}
  \Hypo{b} \Alter{\color{black}} \Infer2[$\otimes$]{c} \Alter{\color{blue}}
  \Hypo{a} \Hypo{b} \Infer2{cccccccccc}
  \Hypo{a} \Hypo{b} \Hypo{bbb} \Infer2{x} \Infer[leftlabel=thing]2{cccccccccc}
  \Alter{\color{red}} \Delims{\left<}{\right>_{i\in\mathbf{N}}}
  \Hypo{foo} \Infer1{b} \Hypo{bbb} \Infer2{x} \Hypo{a}
  \Infer2[$ax^2+bx+c$]{ccccccccc}
  \Delims{\left.}{\right\}\pi_4}
  \Infer[rulestyle=double]4[foo]{ddd}
\end{prooftree}
\end{example}

\subsection{Playing with upside-down}

\begin{example}
\makeatletter
\begin{prooftree}[template=$#1$]
  \Hypo{1} \Hypo{2} \Infer2{3} \Hypo{4} \Infer1{5} \Infer0{6} \Infer3{7}
  \Infer1{A} \Hypo{B} \Infer2{C} \Hypo{D} \Hypo{E} \Infer2{F}
  \ebproof@joinh
  \setkeys{ebproof}{updown}
  \Hypo{a} \Hypo{b} \Infer2{c}
  \Hypo{d} \Hypo{e} \Infer2{f}
  \Hypo{g} \Hypo{h} \Infer2{i}
  \ebproof@joinh
  \ebproof@joinh
  \setkeys{ebproof}{updown=false}
  \ebproof@joinv
\end{prooftree}
\end{example}

\end{document}
