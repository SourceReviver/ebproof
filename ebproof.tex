%% ebproof.sty
%% Copyright 2015 Emmanuel Beffara <manu@beffara.org>
%
% This work may be distributed and/or modified under the
% conditions of the LaTeX Project Public License, either version 1.3
% of this license or (at your option) any later version.
% The latest version of this license is in
%   http://www.latex-project.org/lppl.txt
% and version 1.3 or later is part of all distributions of LaTeX
% version 2005/12/01 or later.
%
% This work has the LPPL maintenance status `maintained'.
%
% The Current Maintainer of this work is Emmanuel Beffara.
%
% This work consists of the files ebproof.sty and ebproof.tex.

\documentclass{article}

\DeclareRobustCommand\package[1]{\texttt{#1}}

\title{The \package{ebproof} package}
\author{Emmanuel Beffara \\ \url{manu@beffara.org}}
\date{Version 1.1 \\ March 13, 2015}

\usepackage{amssymb}
\usepackage{color}
\usepackage{fancyvrb}
\usepackage[a4paper]{geometry}
\usepackage[colorlinks=true,urlcolor=blue]{hyperref}
\usepackage{multicol}

\usepackage{tikz}
\usetikzlibrary{decorations.pathmorphing}

\usepackage{ebproof}

\newcommand\lit[1]{\texttt{#1}}
\newcommand\cs[1]{\lit{\char`\\#1}}
\newcommand\env[1]{\lit{#1}}
\newcommand\opt[1]{\lit{#1}}
\newcommand\meta[1]{$\langle$\textit{#1}$\rangle$}
\newcommand\oarg[1]{\lit{[}\meta{#1}\lit{]}}
\newcommand\marg[1]{\lit{\{}\meta{#1}\lit{\}}}

\newenvironment{csdoc}[1]{%
  \smallbreak\noindent{#1}%
  \begin{list}{}{%
    \topsep=1ex%
  }%
  \item
}{%
  \end{list}%
}

\newenvironment{example}[1]{%
  \def\exampleoptions{#1}%
  \VerbatimOut{example.tex}}{%
  \endVerbatimOut
  \begin{center}
  \begin{minipage}{.4\textwidth}
    \input{example.tex}
  \end{minipage}%
  \begin{minipage}{.6\textwidth}
    \small
    \expandafter\VerbatimInput\expandafter[\exampleoptions]{example.tex}
  \end{minipage}%
  \end{center}
}

\setcounter{tocdepth}{2}

\begin{document}
\maketitle

\begin{multicols}{2}
  \tableofcontents
\end{multicols}

\section{Introduction}

The \env{ebproof} package provides commands to typeset proof trees, in the
style of sequent calculus and related systems:

\begin{example}{gobble=2}
  \begin{prooftree}
    \Hypo{ \Gamma, A &\vdash B }
    \Infer1[abs]{ \Gamma &\vdash A\to B }
    \Hypo{ \Gamma \vdash A }
    \Infer2[app]{ \Gamma \vdash B }
  \end{prooftree}
\end{example}

The structure is very much inspired by the
\href{http://math.ucsd.edu/~sbuss/ResearchWeb/bussproofs/}{\package{bussproofs}}
package, in particular for the postfix notation.
I actually wrote \package{ebproof} because there were some limitations in
\package{bussproofs} that I did not know how to lift, and also because I did
not like some choices in that package (and also because it was fun to write).

\section{Environments}

The package provides the \env{prooftree} environment, in a standard and
starred variants.

\begin{csdoc}{
    \cs{begin}\lit{\{prooftree\}}\oarg{options}
      \meta{statements}
    \cs{end}\lit{\{prooftree\}}}
  Typeset the proof tree desribed by the \meta{statements}, as described in
  section~\ref{sec:statements}.
  The \meta{options} provide default formatting options for the proof tree.
  This environment can be used either in math mode or in text mode.
  It produces a proof tree at the current position in the text flow.
\end{csdoc}

\begin{csdoc}{
    \cs{begin}\lit{\{prooftree*\}}\oarg{options}
      \meta{statements}
    \cs{end}\lit{\{prooftree*\}}}
  Typeset the proof centered on a line of its own; it is essentially
  equivalent to wrapping the \env{prooftree} environment inside a \env{center}
  environment.
\end{csdoc}

\noindent
The starred version is used in situations when a single proof will be
displayed.
The non-starred version is useful in order to integrate the proof into some
larger structure, like two parts of a formula:

\begin{example}{gobble=2}
  \[
    \begin{prooftree}
      \Hypo{ \vdash A }
      \Hypo{ \vdash B } \Infer1{ \vdash B, C }
      \Infer2{ \vdash A\wedge B, C }
    \end{prooftree}
    \quad \rightsquigarrow \quad
    \begin{prooftree}
      \Hypo{ \vdash A } \Hypo{ \vdash B }
      \Infer2{ \vdash A\wedge B }
      \Infer1{ \vdash A\wedge B, C }
    \end{prooftree}
  \]
\end{example}

\section{Statements}
\label{sec:statements}

Statements describes proofs in postfix notation: when typesetting a proof tree
whose last rule has, say, two premisses, you will first write statements for
the subtree of the first premiss, then statements for the subtree of the
second premiss, then a statement like \cs{Infer2}\{\meta{conclusion}\} to
build an inference with these two subtrees as premisses and the given text as
conclusion.

Hence statements operate on a stack of proof trees.
At the beginning of a \lit{prooftree} environment, the stack is empty.
At the end, it must contain exactly one tree, which is the one that will be
printed.

\subsection{Basic statements}

The basic statements for building proofs are the following, where
\meta{options} stands for arbitrary options as described in
section~\ref{sec:options}.
\begin{csdoc}{\cs{Hypo}\oarg{options}\marg{text}}
  Push a new proof tree consisting only in one conclusion line, with no
  premiss and no line above, in other words a tree with only a leaf
  (\cs{Hypo} stands for \emph{hypothesis}).
\end{csdoc}
\begin{csdoc}{\cs{Infer}\oarg{options}\marg{arity}\oarg{label}\marg{text}}
  Build an inference step by taking some proof trees from the top of the
  stack, assembling them with a rule joining their conclusions and putting a
  new conclusion below.
  The \meta{arity} is the number of sub-proofs, it may be any number
  including 0 (in this case there will be a line above the conclusion but no
  sub-proof).
  If \meta{label} is present, it is used as the label on the right of the
  inference line; it is equivalent to using the \opt{right label} option.
\end{csdoc}

The \meta{text} in these statements is the contents of the conclusion at the
root of the tree that the statements create.
It is typeset in math mode by default but any kind of formatting can be used
instead, using the \opt{template} option.
The \meta{label} text is formatted in horizontal text mode by default.

Each proof tree has a vertical axis, used for alignment of successive steps.
The position of the axis is deduced from the text of the conclusion at the
root of the tree: if \meta{text} contains the alignment character \verb|&|
then the axis is set at that position, otherwise the axis is set at the center
of the conclusion text.
The \cs{Infer} statement makes sure that the axis of the premiss is at the
same position as the axis of the conclusion.
If there are several premisses, it places the axis at the center between the
left of the leftmost conclusion and the right of the rightmost conclusion:

\begin{example}{gobble=2}
  \begin{prooftree}
    \Hypo{ &\vdash A, B, C }
    \Infer1{ A &\vdash B, C }
    \Infer1{ A, B &\vdash C }
    \Hypo{ D &\vdash E }
    \Infer2{ A, B, D &\vdash C, E }
    \Infer1{ A, B &\vdash C, D, E }
    \Infer1{ A &\vdash B, C, D, E }
  \end{prooftree}
\end{example}

\subsection{Additional statements}

The following additional statements may be used to affect the format of the
last proof tree on the stack:

\begin{csdoc}{\cs{Ellipsis}\marg{label}\marg{text}}
  Typeset vertical dots, with a label on the right, and a new conclusion.
  No inference lines are inserted.
  \begin{example}{gobble=4}
    \begin{prooftree}
      \Hypo{ \Gamma &\vdash A }
      \Ellipsis{foo}{ \Gamma &\vdash A, B }
    \end{prooftree}
  \end{example}
\end{csdoc}

\begin{csdoc}{\cs{Alter}\marg{code}}
  Modify the proof with arbitrary commands, assuming that these commands do
  not affect the size.
  The \meta{code} is executed in an \cs{hbox} and is followed by the insertion
  of the actual box with the current sub-proof.
  It is mostly useful with \cs{color} commands:
  \begin{example}{gobble=4}
    \begin{prooftree}
      \Hypo{ \Gamma, A &\vdash B }
      \Infer1[abs]{ \Gamma &\vdash A\to B }
      \Alter{\color{red}}
      \Hypo{ \Gamma \vdash A }
      \Infer2[app]{ \Gamma \vdash B }
    \end{prooftree}
  \end{example}
\end{csdoc}

\begin{csdoc}{\cs{Delims}\marg{left}\marg{right}}
  Put left and right delimiters around the whole sub-proof, without changing
  the alignment (the spacing is affected by the delimiters, however).
  The \meta{left} text must contain an opening occurrence of \cs{left} and the
  \meta{right} text must contain a matching occurrence of \cs{right}.
  For instance, \verb|\Delims{\left(}{\right)}| will put the
  sub-proof between parentheses.
  \begin{example}{gobble=4}
    \begin{prooftree}
      \Hypo{ A_1 \vee \cdots \vee A_n }
      \Hypo{ [A_i] }
      \Ellipsis{}{ B }
      \Delims{ \left( }{ \right)_{1\leq i\leq n} }
      \Infer2{ B }
    \end{prooftree}
  \end{example}
\end{csdoc}

\section{Options}
\label{sec:options}

The formatting of trees, conclusion texts and inference rules is affected by
options, specfied using the \LaTeX3 key-value system.
All options are in the \lit{ebproof} module in the key tree.
They can be set locally for a proof tree or for a single statement using
optional arguments in the associated commands.

\begin{csdoc}{\cs{ebproofset}\marg{options}}
  Set some options.
  The options will apply in the current scope; using this in preamble will
  effectively set options globally.
  Specific options may also be specified for each proof tree and for each
  statement in a proof tree, using optional arguments.
\end{csdoc}

\subsection{General shape}

The options in this section only make sense at the global level and at the
proof level.
Changing the proof style inside a \lit{proof} environment has undefined
behaviour.

\begin{csdoc}{\opt{proof style=}\meta{name}}
  Set the general shape for representing proofs.
  The following styles are provided:
  \begin{csdoc}{\lit{upwards}}
    This is the default style.
    Proof trees grow upwards, with conclusions below and premisses above.
  \end{csdoc}
  \begin{csdoc}{\lit{downwards}}
    Proof trees grow downwards, with conclusions above and premisses below.
    \begin{example}{gobble=4}
      \begin{prooftree}[proof style=downwards]
        \Hypo{ \Gamma, A &\vdash B }
        \Infer1[abs]{ \Gamma &\vdash A\to B }
        \Hypo{ \Gamma \vdash A }
        \Infer2[app]{ \Gamma \vdash B }
      \end{prooftree}
    \end{example}
  \end{csdoc}
\end{csdoc}

In the optional argument of \lit{prooftree} environments, proof styles can
be specified directly, without prefixing the name by ``\lit{proof style=}''.
For instance, the first line of the example above could be written
\cs{begin}\lit{\{prooftree\}[downwards]} equivalently.

\begin{csdoc}{\opt{center=}\meta{boolean}}
  If set to \lit{true}, the tree produced by the \env{prooftree} environment
  will be vertically centered around the text line.
  If set to \lit{false}, the base line of the tree will be the base line of
  the conclusion.
  The default value is \lit{true}.
  \begin{example}{gobble=4}
    \begin{prooftree}[center=false]
      \Infer0{ A \vdash A }
    \end{prooftree}
    \qquad
    \begin{prooftree}[center=false]
      \Hypo{ \Gamma, A \vdash B }
      \Infer1{ \Gamma \vdash A \to B }
    \end{prooftree}
  \end{example}
\end{csdoc}

\subsection{Spacing}

\begin{csdoc}{\opt{separation=}\meta{dimension}}
  The horizontal separation between sub-proofs in an inference.
  The default value is \lit{1.5em}.
  \begin{example}{gobble=4}
    \begin{prooftree}[separation=0.5em]
      \Hypo{ A } \Hypo{ B } \Infer2{ C }
      \Hypo{ D } \Hypo{ E } \Hypo{ F } \Infer3{ G }
      \Hypo{ H } \Infer[separation=3em]3{ K }
    \end{prooftree}
  \end{example}
\end{csdoc}

\begin{csdoc}{\opt{rule margin=}\meta{dimension}}
  The spacing above and below inference lines.
  The default value is \lit{0.7ex}.
  \begin{example}{gobble=4}
    \begin{prooftree}[rule margin=2ex]
      \Hypo{ \Gamma, A &\vdash B }
      \Infer1[abs]{ \Gamma &\vdash A\to B }
      \Hypo{ \Gamma \vdash A }
      \Infer2[app]{ \Gamma \vdash B }
    \end{prooftree}
  \end{example}
\end{csdoc}

\subsection{Shape of inference lines}

\begin{csdoc}{\opt{rule style=}\meta{name}}
  Set the shape of inference lines.
  The following values are provided:
  \begin{csdoc}{\opt{simple}}
    A simple horizontal rule is drawn.
    This is the default style.
  \end{csdoc}
  \begin{csdoc}{\opt{no rule}}
    No inference line is drawn.
    A single space of the length of \lit{rule margin} is inserted.
  \end{csdoc}
  \begin{csdoc}{\opt{double}}
    A double line is drawn.
  \end{csdoc}
  \begin{csdoc}{\opt{dashed}}
    A single dashed line is drawn.
  \end{csdoc}
  The precise rendering is influenced by parameters specified below.
  Arbitrary new shapes can defined using the \lit{rule code} option described
  afterwards.
\end{csdoc}

In the optional argument of the \cs{Infer} statement, rule styles can be
specified directly, without prefixing the style name by ``\lit{rule style=}''.
For instance, \cs{Infer}\lit{[dashed]} is equivalent to
\cs{Infer}\lit{[rule style=dashed]}.

\begin{example}{gobble=2}
  \begin{prooftree}
    \Hypo{ \Gamma &\vdash A \to B }
    \Infer[no rule]1{ \Gamma &\vdash {!A} \multimap B }
    \Hypo{ \Delta &\vdash A }
    \Infer[rule thickness=2pt]1{ \Delta &\vdash {!A} }
    \Infer0{ B \vdash B }
    \Infer[dashed]2{ \Delta, {!A}\multimap B \vdash B }
    \Infer2{ \Gamma, \Delta &\vdash B }
    \Infer[double]1{ \Gamma \cup \Delta &\vdash B }
  \end{prooftree}
\end{example}

\begin{csdoc}{\opt{rule thickness=}\meta{dimension}}
  The thickness of inference lines.
  It is \lit{0.4pt} by default.
\end{csdoc}
\begin{csdoc}{\opt{rule separation=}\meta{dimension}}
  The distance between the two lines in the \lit{double} rule style.
  It is \lit{2pt} by default.
\end{csdoc}
\begin{csdoc}{\opt{rule dash length=}\meta{dimension}}
  The length of dashes in the \lit{dashed} rule style.
  It is \lit{0.2em} by default.
\end{csdoc}
\begin{csdoc}{\opt{rule dash space=}\meta{dimension}}
  The space between dashes in the \lit{dashed} rule style.
  It is \lit{0.3em} by default.
\end{csdoc}

\begin{csdoc}{\opt{rule code=}\meta{code}}
  This option is used to define an arbitrary shape for rules.
  The \meta{code} is used to render the rule, it is executed in vertical mode
  in a \cs{vbox} whose \cs{hsize} is set to the width of the rule.
  Margins above and below are inserted automatically (they can be removed by
  setting \lit{rule margin} to \lit{0pt}).

  \begin{example}{gobble=4}
    \begin{prooftree}[rule code={\hbox{\tikz
        \draw[decorate,decoration={snake,amplitude=.3ex}]
        (0,0) -- (\hsize,0);}}]
      \Hypo{ \Gamma &\vdash A }
      \Infer1{ \Gamma &\vdash A, \ldots, A }
      \Hypo{ \Delta, A, \ldots, A \vdash \Theta }
      \Infer2{ \Gamma, \Delta \vdash \Theta }
    \end{prooftree}
  \end{example}
  Note that this example requires the \package{tikz} package, with the
  \package{decorations.pathmorphing} library for the \lit{snake} decoration.
\end{csdoc}

The option \opt{rule code} is particularly useful in a ``meta-key'' in the
sense of \LaTeX3 keys as it allows to define new rule styles.
The allowed values for \opt{rule style} are actually defined this way.
The above example could be turned into a new rule style \lit{zigzag} with the
following command:
\begin{example}{gobble=2}
  \ExplSyntaxOn
  \keys_define:nn { ebproof / rule~style } {
    zigzag .meta:n = { rule~code = { \hbox{\tikz
      \draw[decorate,decoration={snake,amplitude=.3ex}]
      (0,0) -- (\hsize,0);} }}}
  \ExplSyntaxOff
  \begin{prooftree}
    \Hypo{ \Gamma &\vdash A }
    \Infer1{ \Gamma &\vdash A, \ldots, A }
    \Hypo{ \Delta, A, \ldots, A \vdash \Theta }
    \Infer[zigzag]2{ \Gamma, \Delta \vdash \Theta }
  \end{prooftree}
\end{example}

\subsection{Format of conclusions}

\begin{csdoc}{%
    \opt{template=}\meta{code} \\
    \opt{left template=}\meta{code} \\
    \opt{right template=}\meta{code}}
  Defines how conclusions are formatted.
  The code is arbitrary \TeX\ code, composed in horizontal mode.
  The macro \cs{inserttext} can be used inside the actual text passed to the
  \cs{Hypo} and \cs{Infer} statements.
  The \opt{template} value is used for conclusions with no alignment mark.
  The \opt{left template} and \opt{right template} values are used on the left
  and right side of the alignment mark when it is present.
  The default value for \opt{template} is simply \verb|$\inserttext$|, so that
  conclusions are set in math mode.
  The default values for \opt{left template} and \opt{right template} are
  similar, with spacing assuming that a relation symbol is put near the
  alignment mark, so that \verb|\Infer1{A &\vdash B}| is spaced correctly.

  \begin{example}{gobble=4}
    \begin{prooftree}[template=(\textbf\inserttext)]
      \Hypo{ foo }
      \Hypo{ bar }
      \Infer1{ baz }
      \Infer2{ quux }
    \end{prooftree}
  \end{example}
\end{csdoc}

\subsection{Format of labels}

\begin{csdoc}{%
    \opt{left label=}\meta{text} \\
    \opt{right label=}\meta{text}}
  The text to use as the labels of the rules, on the left and on the right
  of the inference line.
  Using the second optional argument in \cs{Infer} is equivalent to setting
  the \env{right label} option with the value of that argument.
\end{csdoc}
\begin{csdoc}{%
    \opt{left label template=}\meta{code} \\
    \opt{right label template=}\meta{code}}
  Defines how rule labels are formatted.
  The code is arbitrary \TeX\ code, composed in horizontal mode.
  The macro \cs{inserttext} can be used to insert the actual label text, as
  defined by the options above.
  The default values are simply \cs{inserttext} so that labels are set in
  plain text mode.
\end{csdoc}

\begin{csdoc}{\opt{label separation=}\meta{dimension}}
  The spacing between an inference lines and its labels.
  The default value is \lit{0.5em}.
\end{csdoc}


\section{License}

This work may be distributed and/or modified under the
conditions of the \LaTeX\ Project Public License, either version 1.3
of this license or (at your option) any later version.
The latest version of this license is in
\begin{center}
  \url{http://www.latex-project.org/lppl.txt}
\end{center}
and version 1.3 or later is part of all distributions of \LaTeX\
version 2005/12/01 or later.

This work has the LPPL maintenance status `maintained'.

The Current Maintainer of this work is Emmanuel Beffara.

This work consists of the files \texttt{ebproof.sty} and \texttt{ebproof.tex}.

\end{document}
