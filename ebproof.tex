\documentclass{article}

\DeclareRobustCommand\package[1]{\texttt{#1}}

\title{\package{ebproof} -- proof trees}
\author{Emmanuel Beffara}

\usepackage{amssymb}
\usepackage{color}
\usepackage{fancyvrb}
\usepackage[a4paper]{geometry}
\usepackage{hyperref}
\usepackage{multicol}

\usepackage{tikz}
\usetikzlibrary{decorations.pathmorphing}

\usepackage{ebproof}

\newcommand\lit[1]{\texttt{#1}}
\newcommand\cs[1]{\lit{\char`\\#1}}
\newcommand\env[1]{\lit{#1}}
\newcommand\opt[1]{\lit{#1}}
\newcommand\meta[1]{$\langle$\textit{#1}$\rangle$}
\newcommand\oarg[1]{\lit{[}\meta{#1}\lit{]}}
\newcommand\marg[1]{\lit{\{}\meta{#1}\lit{\}}}

\newenvironment{csdoc}[1]{%
  \smallbreak\noindent{#1}%
  \begin{list}{}{%
    \topsep=1ex%
%     \leftmargin=\parindent%
  }%
  \item
}{%
  \end{list}%
}

\newenvironment{example}{%
  \VerbatimOut{example.tex}}{%
  \endVerbatimOut
  \begin{center}
  \begin{minipage}{.4\textwidth}
    \input{example.tex}
  \end{minipage}%
  \begin{minipage}{.6\textwidth}
    \VerbatimInput{example.tex}
  \end{minipage}%
  \end{center}
}

\ebproofset{zigzag/.style={rule code={\hbox{%
  \tikz\draw[decorate,decoration={snake,amplitude=.3ex}]
    (0,0) -- (\hsize,0);}}}}

\begin{document}
\maketitle

\tableofcontents

\section{Introduction}

The \env{ebproof} package provides commands to typeset proof trees, in the
style of sequent calculus and related systems:

\begin{example}
\begin{prooftree}
  \Hypo{ \Gamma, A &\vdash B }
  \Infer1{ \Gamma &\vdash A\to B }
  \Hypo{ \Gamma \vdash A }
  \Infer2{ \Gamma \vdash B }
\end{prooftree}
\end{example}

The structure is very much inspired by the
\href{http://math.ucsd.edu/~sbuss/ResearchWeb/bussproofs/}{\package{bussproofs}}
package, in particular for the postfix notation.
I actually wrote \package{ebproof} because there were some limitations in
\package{bussproofs} that I did not know how to lift, and also because I did
not like some choices in that package (and also because it was fun to write).

\section{Usage}

Proofs are built using statements that operate on a stack of sub-proofs.
The main statements \cs{Hypo} and \cs{Infer} (see
section~\ref{sec:statements}) are used to create proofs and assemble them in
inference trees.
Other statements and various options are used to fine-tune the presentation of
trees.

\subsection{Environments}

The package provides the \env{prooftree} environment, in a standard and
starred variants.

\begin{csdoc}{
    \cs{begin}\lit{\{prooftree\}}\oarg{options}
      \meta{statements}
    \cs{end}\lit{\{prooftree\}}}
  Typeset the proof tree desribed by the \meta{statements}, as described in
  section~\ref{sec:statements}.
  The \meta{options} provide default formatting options for the proof tree.
  This environment can be used either in math mode or in text mode.
  It produces a proof tree at the current position in the text flow.
\end{csdoc}

\begin{csdoc}{
    \cs{begin}\lit{\{prooftree*\}}\oarg{options}
      \meta{statements}
    \cs{end}\lit{\{prooftree*\}}}
  Typeset the proof centered on a line of its own; it is essentially
  equivalent to wrapping the \env{prooftree} environment inside a \env{center}
  environment.
\end{csdoc}

\noindent
The starred version is used in situations when a single proof will be
displayed.
The non-starred version is useful in order to integrate the proof into some
larger structure, like two parts of a formula:

\begin{example}
\[
  \begin{prooftree}
    \Hypo{ \vdash A }
    \Hypo{ \vdash B }
    \Infer1{ \vdash B, C }
    \Infer2{ \vdash A\wedge B, C }
  \end{prooftree}
  \quad \rightsquigarrow \quad
  \begin{prooftree}
    \Hypo{ \vdash A }
    \Hypo{ \vdash B }
    \Infer2{ \vdash A\wedge B }
    \Infer1{ \vdash A\wedge B, C }
  \end{prooftree}
\]
\end{example}

% \begin{example}
% \begin{prooftree}
%   \Hypo{\strut \Pi_1 }
%   \Infer[rulestyle=none]1{ A }
%   \Hypo{ [A] }
%   \Infer[rulestyle=none]1{ \Pi_2 }
%   \Infer[rulestyle=none]1{ B }
%   \Infer1{ A\to B }
%   \Infer2{ B }
% \end{prooftree}
% \end{example}

\subsection{Statements}
\label{sec:statements}

Statements describes proofs in postfix notation: when typesetting a proof tree
whose last rule has, say, two premisses, you will first write statements for
the subtree of the first premiss, then statements for the subtree of the
second premiss, then a statement like \cs{Infer2}\{\meta{conclusion}\} to
build an inference with these two subtrees as premisses and the given text as
conclusion.

Hence statements operate on a stack of proof trees.
At the beginning of a \lit{prooftree} environment, the stack is empty.
At the end, it must contain exactly one tree, which is the one that will be
printed.

\subsubsection{Basic statements}

The basic statements for building proofs are the following, where
\meta{options} stands for arbitrary options as described in
section~\ref{sec:options}.
\begin{csdoc}{\cs{Hypo}\oarg{options}\marg{text}}
  Push a new proof tree consisting only in one conclusion line, with no
  premiss and no line above, in other words a tree with only a leaf
  (\cs{Hypo} stands for \emph{hypothesis}).
\end{csdoc}
\begin{csdoc}{\cs{Infer}\oarg{options}\marg{arity}\oarg{label}\marg{text}}
  Build an inference step by taking some proof trees from the top of the
  stack, assembling them with a rule joining their conclusions and putting a
  new conclusion below.
  The \meta{arity} is the number of sub-proofs, it may be any number
  including 0 (in this case there will be a line above the conclusion but no
  sub-proof).
  If \meta{label} is present, it is used as the label on the right of the
  inference line; it is equivalent to using the \opt{rightlabel} option.
\end{csdoc}

The \meta{text} in these statements is the contents of the conclusion at the
root of the tree that the statements create.
It is typeset in math mode by default but any kind of formatting can be used
instead, using the \opt{template} option.
The \meta{label} text is formatted in horizontal text mode by default.

\begin{example}
\begin{prooftree}[template=(\textbf{#1})]
  \Hypo{ foo }
  \Hypo{ bar }
  \Infer1{ baz }
  \Infer2{ quux }
\end{prooftree}
\end{example}

Each proof tree has a vertical axis, used for alignment of successive steps.
The position of the axis is deduced from the text of the conclusion at the
root of the tree: if \meta{text} contains the alignment character \verb|&|
then the axis is set at that position, otherwise the axis is set at the center
of the conclusion text.
The \cs{Infer} statement makes sure that the axis of the premiss is at the
same position as the axis of the conclusion.
If there are several premisses, it places the axis at the center between the
left of the leftmost conclusion and the right of the rightmost conclusion:

\begin{example}
\begin{prooftree}
  \Hypo{ &\vdash A, B, C }
  \Infer1{ A &\vdash B, C }
  \Infer1{ A, B &\vdash C }
  \Hypo{ D &\vdash E }
  \Infer2{ A, B, D &\vdash C, E }
  \Infer1{ A, B &\vdash C, D, E }
  \Infer1{ A &\vdash B, C, D, E }
\end{prooftree}
\end{example}

\subsubsection{Additional statements}

The following additional statements may be used to affect the format of the
last proof tree on the stack:

\begin{csdoc}{\cs{Ellipsis}\marg{label}\marg{text}}
  Typeset vertical dots, with a label on the right, and a new conclusion.
  No inference lines are inserted.
  \begin{example}
    \begin{prooftree}
      \Hypo{ \Gamma &\vdash A }
      \Ellipsis{foo}{ \Gamma &\vdash A, B }
    \end{prooftree}
  \end{example}
\end{csdoc}
\begin{csdoc}{\cs{Alter}\marg{commands}}
  Modify the proof with arbitrary commands, assuming that these commands do
  not affect the size.
  A typical use is \verb|\Alter{\color{red}}|, this will typeset the subproof
  in red.
\end{csdoc}
\begin{csdoc}{\cs{Delims}\marg{left}\marg{right}}
  Put left and right delimiters around the whole sub-proof, without changing
  the alignment (the spacing is affected by the delimiters, however).
  The \meta{left} text must contain an opening occurrence of \cs{left} and the
  \meta{right} text must contain a matching occurrence of \cs{right}.
  For instance, \verb|\Delims{\left(}{\right)}| will put the
  sub-proof between parentheses.
\end{csdoc}

\subsection{Options}
\label{sec:options}

The formatting of trees, conclusion texts and inference rules is affected by
options, specfied using the key-value system of PGF/TikZ, provided by the
\package{pgfkeys} package.
All options are in the \lit{/ebproof/} path in the key tree of
\package{pgfkeys}.
They can be set locally for a tree or a single statement using optional
arguments.

\begin{csdoc}{\cs{ebproofset}\marg{options}}
  Set some options.
  The options will apply in the current scope; using this in preamble will
  effectively set options globally.
  Specific options may also be specified for each proof tree and for each
  statement in a proof tree, using optional arguments.
\end{csdoc}

\subsubsection{General shape}

\begin{csdoc}{\opt{updown=}\meta{boolean}}
  If set to \lit{true}, the proofs are written upside down, with the
  conclusion of rules set above and the premisses below.
\end{csdoc}
\begin{csdoc}{\opt{center=}\meta{boolean}}
  If set to \lit{true}, the tree produced by the \env{prooftree} environment
  will be vertically centered around the text line.
  If set to \lit{false}, the base line of the tree will be the base line of
  the conclusion.
  The default value is \lit{true}.
\end{csdoc}

\subsubsection{Shape of inference lines}

\begin{csdoc}{\opt{simple}}
  A simple horizontal rule is drawn.
  This is the default style.
\end{csdoc}
\begin{csdoc}{\opt{none}}
  No inference line is drawn at all.
  A single space of the length of \lit{rulemargin} is inserted.
\end{csdoc}
\begin{csdoc}{\opt{double}}
  A double line is drawn.
  The spacing between lines is given by the \lit{double/sep} option.
\end{csdoc}
\begin{csdoc}{\opt{dashed}}
  A single dashed line is drawn.
  The length of dashes is given by the \lit{double/on} option, the spacing
  between dashes is given by \lit{double/off}.
\end{csdoc}

\begin{example}
  \begin{prooftree}
    \Hypo{ \Gamma &\vdash A \to B }
    \Infer[none]1{ \Gamma &\vdash {!A} \multimap B }
    \Hypo{ \Delta &\vdash A }
    \Infer1{ \Delta &\vdash {!A} }
    \Infer0{ B \vdash B }
    \Infer[dashed]2{ \Delta, {!A}\multimap B \vdash B }
    \Infer2{ \Gamma, \Delta &\vdash B }
    \Infer[double]1{ \Gamma \cup \Delta &\vdash B }
  \end{prooftree}
\end{example}

\begin{csdoc}{\opt{rule code=}\meta{code}}
  This option is used to define an arbitrary shape for rules.
  The \meta{code} is used to render the rule, it is executed in vertical mode
  in a \cs{vbox} whose \cs{hsize} is set to the width of the rule.
  Margins above and below are inserted automatically (they can be removed by
  setting \lit{rulemargin} to \lit{0pt}).

  This option is particularly useful in a ``styles'' in the sense of
  \package{pgfkeys} (the options above are actually styles that define the
  value of \lit{rule code} to common useful values).
\end{csdoc}

\begin{verbatim}
\usepackage{tikz}
\usetikzlibrary{decorations.pathmorphing}
\ebproofset{zigzag/.style={rule code={\hbox{%
  \tikz\draw[decorate,decoration={snake,amplitude=.3ex}]
    (0,0) -- (\hsize,0);}}}}
\end{verbatim}
\begin{example}
\begin{prooftree}
  \Hypo{ \Gamma &\vdash A }
  \Infer1{ \Gamma &\vdash A, \ldots, A }
  \Hypo{ \Delta, A, \ldots, A \vdash \Theta }
  \Infer[zigzag]2{ \Gamma, \Delta \vdash \Theta }
\end{prooftree}
\end{example}

\subsubsection{Labels}

\begin{csdoc}{%
    \opt{leftlabel=}\meta{text} \\
    \opt{rightlabel=}\meta{text}}
  The text to use as the labels of the rules, on the left and on the right
  of the inference line.
  Using the second optional argument in \cs{Infer} is equivalent to setting
  the \env{rightlabel} option with the value of that argument.
\end{csdoc}
\begin{csdoc}{%
    \opt{template=}\meta{macro} \\
    \opt{lefttemplate=}\meta{macro} \\
    \opt{righttemplate=}\meta{macro}}
  Defines how conclusions are formatted.
  The macros are arbitrary \TeX\ code, composed in horizontal mode, with the
  macro argument \verb|#1| standing for the actual text passed to the
  \cs{Hypo} and \cs{Infer} statements.
  The \opt{template} value is used for conclusions with no alignment mark.
  The \opt{lefttemplate} and \opt{righttemplate} values are used on the left
  and right side of the alignment mark when it is present.
  The default value for \opt{template} is simply \verb|$#1$|, so that
  conclusions are set in math mode.
  The default values for \opt{lefttemplate} and \opt{righttemplate} are
  similar, with spacing assuming that a relation symbol is put near the
  alignment mark, so that \verb|\Infer1{A &\vdash B}| is spaced correctly.
\end{csdoc}
\begin{csdoc}{%
    \opt{leftlabeltemplate=}\meta{macro} \\
    \opt{rightlabeltemplate=}\meta{macro}}
  These macros are used to typeset the text of labels on the left and right of
  inference lines.
  The default values are \verb|#1| so that labels are set in plain text mode.
\end{csdoc}

\subsubsection{Dimensions}

\begin{csdoc}{\opt{hsep=}\meta{dimension}}
  The horizontal separation between sub-proofs in an inference.
\end{csdoc}
\begin{csdoc}{\opt{rulemargin=}\meta{dimension}}
  The spacing above and below inference lines.
\end{csdoc}
\begin{csdoc}{\opt{thickness=}\meta{dimension}}
  The thickness of inference lines.
\end{csdoc}
\begin{csdoc}{\opt{labelsep=}\meta{dimension}}
  The spacing between an inference lines and its labels.
\end{csdoc}

\section{Examples}

\subsection{Simple proof tree}

\begin{example}
\begin{prooftree}
  \Hypo{ \Gamma, A &\vdash B }
  \Infer1{ \Gamma &\vdash A\to B }
  \Hypo{ \Gamma \vdash A }
  \Infer2{ \Gamma \vdash B }
\end{prooftree}
\end{example}

\subsection{Templates and alignment}

\begin{example}
\begin{prooftree}[dashed,righttemplate={,#1}]
  \Hypo{ABC}
  \Infer1{&ABC}
  \Infer1{A&BC}
  \Infer1{AB&C}
  \Infer1{ABC&}
\end{prooftree}
\end{example}

\subsection{Fancy formatting}

\begin{example}
\begin{prooftree}
  \Hypo{A}
  \Ellipsis{$\pi_1$}{B}
  \Hypo{b} \Alter{\color{black}} \Infer2[$\otimes$]{c} \Alter{\color{blue}}
  \Hypo{a} \Hypo{b} \Infer2{cccccccccc}
  \Hypo{a} \Hypo{b} \Hypo{bbb} \Infer2{x} \Infer[leftlabel=thing]2{cccccccccc}
  \Alter{\color{red}} \Delims{\left<}{\right>_{i\in\mathbf{N}}}
  \Hypo{foo} \Infer1{b} \Hypo{bbb} \Infer2{x} \Hypo{a}
  \Infer2[$ax^2+bx+c$]{ccccccccc}
  \Delims{\left.}{\right\}\pi_4}
  \Infer[double]4[foo]{ddd}
\end{prooftree}
\end{example}

\subsection{Playing with upside-down}

\begin{example}
\makeatletter
\begin{prooftree}[template=$#1$]
  \Hypo{1} \Hypo{2} \Infer2{3} \Hypo{4} \Infer1{5} \Infer0{6} \Infer3{7}
  \Infer1{A} \Hypo{B} \Infer2{C} \Hypo{D} \Hypo{E} \Infer2{F}
  \ebproof@joinh
  \ebproofset{updown}
  \Hypo{a} \Hypo{b} \Infer2{c}
  \Hypo{d} \Hypo{e} \Infer2{f}
  \Hypo{g} \Hypo{h} \Infer2{i}
  \ebproof@joinh
  \ebproof@joinh
  \ebproofset{updown=false}
  \ebproof@joinv
\end{prooftree}
\end{example}

\end{document}
